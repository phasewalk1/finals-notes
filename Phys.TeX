%%%%%%%%%%%%%%%%%%%%%%%%%%%%%%%%%%%%%%%%%%%%%%%%%%%%%%%%%%%%%%%%%%%%%%
% LaTeX Template: Two Column Colour Article
%
% Source: http://www.howtotex.com/
% Feel free to distribute this template, but please keep the
% referal to howtotex.com.
% Date: Feb 2011
% 
%%%%%%%%%%%%%%%%%%%%%%%%%%%%%%%%%%%%%%%%%%%%%%%%%%%%%%%%%%%%%%%%%%%%%%
% How to use overleaf.com: 
%
% You edit the source code here on the left, and the preview on the
% right shows you the result within a few seconds.
%
% You can upload figures, bibliographies, custom classes and
% styles using the files menu.
%
% If you're new to LaTeX, the wikibook is a great place to start:
% http://en.wikibooks.org/wiki/LaTeX
%
%%%%%%%%%%%%%%%%%%%%%%%%%%%%%%%%%%%%%%%%%%%%%%%%%%%%%%%%%%%%%%%%%%%%%%
% adaptions made by wolfgang stoettner mail@stoettner.net
%%%%%%%%%%%%%%%%%%%%%%%%%%%%%%%%%%%%%%%%%%%%%%%%%%%%%%%%%%%%%%%%%%%%%%

%%% Preamble
\documentclass[	DIV=calc,%
							paper=a4,%
							fontsize=11pt,%
							twocolumn]{scrartcl} % KOMA-article class

\usepackage{lipsum}	% Package to create dummy text
\usepackage[english]{babel}	% English language/hyphenation
\usepackage[protrusion=true,expansion=true]{microtype}	% Better typography
\usepackage{amsmath,amsfonts,amsthm} % Math packages
\usepackage[pdftex]{graphicx} % Enable pdflatex
\usepackage{wrapfig} % enable figure wrapping
\usepackage[svgnames]{xcolor} % Enabling colors by their 'svgnames'
\usepackage[hang, small,labelfont=bf,up,textfont=it,up]{caption} % Custom captions under/above floats
\usepackage{epstopdf} % Converts .eps to .pdf
\usepackage{xcolor, soul}
\usepackage{subfig}	% Subfigures
\usepackage{booktabs} % Nicer tables
\usepackage{fix-cm}	% Custom fontsizes
\usepackage{booktabs} % prof. looking tables (www.en.wikibooks.org/wiki/LaTeX/Tables#Professional_tables)
\usepackage{amssymb}

%%% Custom sectioning (sectsty package)
\usepackage{sectsty} % Custom sectioning (see below)
\allsectionsfont{%		% Change font of al section commands
	\usefont{OT1}{phv}{b}{n}%	% bch-b-n: CharterBT-Bold font
	}

\sectionfont{%		% Change font of \section command
	\usefont{OT1}{phv}{b}{n}%	% bch-b-n: CharterBT-Bold font
	}


%%% Headers and footers
\usepackage{fancyhdr} % Needed to define custom headers/footers
	\pagestyle{fancy} % Enabling the custom headers/footers
\usepackage{lastpage}	

% Header (empty)
\lhead{}
\chead{}
% Footer (you may change this to your own needs)
\lfoot{\footnotesize \texttt{PHYS-129 Final Notes: Anne Gallucci}}
\cfoot{}
\rfoot{\footnotesize page \thepage\ of \pageref{LastPage}}	% "Page 1 of 2"
\renewcommand{\headrulewidth}{0.0pt}
\renewcommand{\footrulewidth}{0.4pt}
\newcommand{\hformbar}[1]{\vspace{5pt}\hrule\vspace{10pt}} % creates a horizontal bar to separate formulae better; space adaptions can be made centrally here

%%% a primed vector
\newcommand{\pvec}[1]{\vec{#1}\mkern2mu\vphantom{#1}}

%%% Creating an initial of the very first character of the content
\usepackage{lettrine}
\newcommand{\initial}[1]{%
     \lettrine[lines=3,lhang=0.3,nindent=0em]{
     				\color{DarkGoldenrod}
     				{\textsf{#1}}}{}}

%%% Title, author and date metadata
\usepackage{titling} % For custom titles

\newcommand{\HorRule}{\color{DarkGoldenrod}%	% Creating a horizontal rule
									  	\rule{\linewidth}{1pt}%
										}

\date{\today} % No date

%%% wws: create a non-indented formula name
\newcommand{\formdesc}[1]{\noindent\textbf{#1}}



%%% Begin document -----------------------------------------------------------------
\begin{document}
\thispagestyle{fancy} 	% Enabling the custom headers/footers for the first page 
\hspace{10pt}

%%% LOGIC & PHILOSOPHY OF SCIENCE STUFF -----------------------------------------------
\sethlcolor{pink}
\formdesc{\hl{Logic, Scientific Philosophy}}

\textbf{Formal Logic Notation}
\begin{enumerate}
    \item $\neg\quad\text{or}\quad\sim$
        Denotes logical \textbf{NOT}
    \item $\land\qquad\qquad$
        Denotes logical \textbf{AND}
    \item $\lor\qquad\qquad$
        Denotes logical \textbf{OR}
    \item $\therefore\qquad\qquad$
        Denotes \textbf{therefore}
    \item $p\rightarrow q$\qquad \  Denotes a \textbf{conditional}
    \item $q\rightarrow p$\qquad \ Denotes a \textbf{converse}
    \item $\neg p\rightarrow\neg q$\quad Denotes a \textbf{inverse}
    \item $\neg q\rightarrow\neg p$\quad Denotes a \textbf{contrapositive}
    \item $p\iff q$\quad Denotes a \textbf{biconditional}
\end{enumerate}

\hformbar{}

%%% JUST VECTOR THINGS ----------------------------------------------------------------

\sethlcolor{pink}
\formdesc{\hl{Vectors}}

The magnitude of a vector:
\begin{equation}
    v = \sqrt{v_x^2 + v_y^2 + ... + v_n^2}
\end{equation}

Finding an angle between two vectors:
\begin{equation}
    \cos\theta = \frac{\vec A \cdot \vec B}{AB}
\end{equation}
\hformbar{}

%%% CONDITIONS ---------------------------------------------------------
\formdesc{\hl{Constraints (Can I use this?)}}

\textbf{Is Energy Conserved?}
\begin{itemize}
    \item Energy is conserved if \textbf{$W_{NC}=0$}
\end{itemize}

\textbf{Is Momentum Conserved?}
\begin{itemize}
    \item Linear momentum is conserved if \textbf{no net external force} acts on the object
    \item Angular Momentum is conserved if \textbf{no external torque }acts on the object
\end{itemize}
\hformbar{}

%%% PROJECTILE MOTION -----------------------------------------------------------------
\formdesc{\hl{Projectile Motion}}
\sethlcolor{red}

\textbf{\hl{ONLY IF}} $h_i\equiv h_f$

Height $\rightarrow$\begin{equation}
    H = \frac{v_0^2\sin^2\theta}{2g}
\end{equation}

Range $\rightarrow$\begin{equation}
     R = \frac{v_0^2\sin{2\theta}}{g}
\end{equation}

Time $\rightarrow$\begin{equation}
    T = \frac{2v_0\sin\theta}{g}
\end{equation}
\hformbar{}

%%% BEGIN MOMENTUM SECTION ------------------------------------------------------------
\sethlcolor{pink}
\formdesc{\hl{Momentum}}

\textbf{Linear}:
\begin{equation}
    \vec p = m\vec v
\end{equation}

\textbf{Angular}:
\begin{equation}
    \vec L = I\vec\omega
\end{equation}

\textbf{2nd Law with Momentum}:
\begin{equation}
    \sum \vec F = \frac{d\vec p}{dt}, \qquad \frac{d\vec p}{dt}(m)+\frac{d}{dt}(\vec v)
\end{equation}

\textbf{Impulse}:
\begin{equation}
    \vec J = \int\vec F \ dt \
\end{equation}

\textbf{Conservation of Momentum}:
\begin{equation}
    m_A\vec v_A + m_B\vec v_B = m_A\pvec{v}'_A+m_B\pvec{v}'_B
\end{equation}
%%% MOMENTUM CONSTRAINTS
\sethlcolor{yellow}

\hl{\textbf{where}}: Collision is quick, i.e., $\sum\vec F_{ext}=0$
\hformbar{}
%%% MOMENTUM TRICKS
\sethlcolor{lime}
\hl{\textbf{Tricks}}
\begin{enumerate}
    \item Where $A$ has $x$ times the mass as $B$ and kinetic after energy $K^\prime_A$, then $K^\prime_B=xK^\prime_A$
\end{enumerate}

\hformbar{}
\hformbar{}
%%% END MOMETUM SECTION ---------------------------------------------------------------

%%% Rotational Dynamics ------------------------------------------------
\sethlcolor{pink}
\formdesc{\hl{Rotational Dynamics}}

\textbf{Torques}
\begin{equation}
    \tau = rF\sin\phi
\end{equation}

\textbf{Work with Rotational Energy}
\begin{equation}
    W=\tau\Delta\theta
\end{equation}

\textbf{Energy of a Rotating Object}
\begin{equation}
    K_{\text{rot}}=\frac{1}{2}I\omega^2
\end{equation}

\textbf{2nd Law with Rotational Dynamics}
\begin{equation}
    \sum\vec\tau=I\vec\alpha,\qquad\alpha=\frac{\tau}{I}
\end{equation}

\textbf{Relating Rotational Variables}
    $$F\iff\tau,\qquad  v\iff\omega$$
    $$a\iff\alpha,\qquad p\iff L$$
\hformbar{}

%%% MISC. SECTION ---------------------------------------------------------------------
\sethlcolor{pink}{\formdesc{\hl{Misc.}}}

\textbf{Hooke's Law}

Where $k$ is the spring's constant and $x$ its displ.:
\begin{equation}
    F_s = -kx
\end{equation}

\textbf{Power}
\begin{enumerate}
    \item Linear\qquad\qquad$P=\vec F\cdot\vec v$
    \item Rotational\qquad$P=\vec\tau\cdot\vec\omega$
\end{enumerate}

\textbf{Collision Types}:
        
        \qquad\textbf{Elastic}: $\sum K = \sum K^\prime$  
        
        \qquad\textbf{Inelastic}: $\sum K\ne\sum K^\prime$
            
            \qquad\qquad Explosive: $\sum K^\prime > \sum K$

            \qquad\qquad Sticky: $\sum K^\prime < \sum K$

            \qquad\qquad Completely inelastic: $\vec p = (m_A+m_B)\pvec{v}$

\textbf{Nonconservative Forces}

\quad The total energy of a system $E_\text{sys}$ is not conserved if nonconservative       forces do work, ($W_{\text{NC}}\ne 0$). Below are examples of common, nonconservative forces
\begin{itemize}
    \item Friction
    \item Drag
    \item Tension
\end{itemize}
        
    \hformbar{}
    \end{document}
