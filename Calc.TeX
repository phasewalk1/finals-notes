%%%%%%%%%%%%%%%%%%%%%%%%%%%%%%%%%%%%%%%%%%%%%%%%%%%%%%%%%%%%%%%%%%%%%%
% LaTeX Template: Two Column Colour Article
%
% Source: http://www.howtotex.com/
% Feel free to distribute this template, but please keep the
% referal to howtotex.com.
% Date: Feb 2011
% 
%%%%%%%%%%%%%%%%%%%%%%%%%%%%%%%%%%%%%%%%%%%%%%%%%%%%%%%%%%%%%%%%%%%%%%
% How to use overleaf.com: 
%
% You edit the source code here on the left, and the preview on the
% right shows you the result within a few seconds.
%
% You can upload figures, bibliographies, custom classes and
% styles using the files menu.
%
% If you're new to LaTeX, the wikibook is a great place to start:
% http://en.wikibooks.org/wiki/LaTeX
%
%%%%%%%%%%%%%%%%%%%%%%%%%%%%%%%%%%%%%%%%%%%%%%%%%%%%%%%%%%%%%%%%%%%%%%
% adaptions made by wolfgang stoettner mail@stoettner.net
%%%%%%%%%%%%%%%%%%%%%%%%%%%%%%%%%%%%%%%%%%%%%%%%%%%%%%%%%%%%%%%%%%%%%%

%%% Preamble
\documentclass[	DIV=calc,%
							paper=a4,%
							fontsize=11pt,%
							twocolumn]{scrartcl} % KOMA-article class

\usepackage{lipsum}	% Package to create dummy text
\usepackage[english]{babel}	% English language/hyphenation
\usepackage[protrusion=true,expansion=true]{microtype}	% Better typography
\usepackage{amsmath,amsfonts,amsthm} % Math packages
\usepackage[pdftex]{graphicx} % Enable pdflatex
\usepackage{wrapfig} % enable figure wrapping
\usepackage[svgnames]{xcolor} % Enabling colors by their 'svgnames'
\usepackage[hang, small,labelfont=bf,up,textfont=it,up]{caption} % Custom captions under/above floats
\usepackage{epstopdf} % Converts .eps to .pdf
\usepackage{subfig}	% Subfigures
\usepackage{booktabs} % Nicer tables
\usepackage{fix-cm}	% Custom fontsizes
\usepackage{booktabs} % prof. looking tables (www.en.wikibooks.org/wiki/LaTeX/Tables#Professional_tables)

%%% Custom sectioning (sectsty package)
\usepackage{sectsty} % Custom sectioning (see below)
\allsectionsfont{%		% Change font of al section commands
	\usefont{OT1}{phv}{b}{n}%	% bch-b-n: CharterBT-Bold font
	}

\sectionfont{%		% Change font of \section command
	\usefont{OT1}{phv}{b}{n}%	% bch-b-n: CharterBT-Bold font
	}


%%% Headers and footers
\usepackage{fancyhdr} % Needed to define custom headers/footers
	\pagestyle{fancy} % Enabling the custom headers/footers
\usepackage{lastpage}	

% Header (empty)
\lhead{}
\chead{}
% Footer (you may change this to your own needs)
\lfoot{\footnotesize \texttt{Calculus 1 Final Notecard}}
\cfoot{}
\rfoot{\footnotesize page \thepage\ of \pageref{LastPage}}	% "Page 1 of 2"
\renewcommand{\headrulewidth}{0.0pt}
\renewcommand{\footrulewidth}{0.4pt}
\newcommand{\hformbar}[1]{\vspace{5pt}\hrule\vspace{10pt}} % creates a horizontal bar to separate formulae better; space adaptions can be made centrally here


%%% Creating an initial of the very first character of the content
\usepackage{lettrine}
\newcommand{\initial}[1]{%
     \lettrine[lines=3,lhang=0.3,nindent=0em]{
     				\color{DarkGoldenrod}
     				{\textsf{#1}}}{}}

%%% Title, author and date metadata
\usepackage{titling} % For custom titles

\newcommand{\HorRule}{\color{DarkGoldenrod}%	% Creating a horizontal rule
									  	\rule{\linewidth}{1pt}%
										}

\date{\today} % No date

%%% wws: create a non-indented formula name
\newcommand{\formdesc}[1]{\noindent\textbf{#1}}



%%% Begin document -----------------------------------------------------------------
\begin{document}
\thispagestyle{fancy} 	% Enabling the custom headers/footers for the first page 
\hspace{10pt}

%%% DEFINITION OF THE DERIVATIVE
\formdesc{Derivative Definition}
\begin{equation}
    \frac{d}{dx}[f(x)] = \lim_{h\rightarrow 0}\frac{f(x+h)-f(x)}{h}
\end{equation}
\hformbar

%%% DIFFERENTATION RULES
\formdesc{Differentiation Rules}
\begin{equation}
    \frac{d}{dx}[f(x)g(x)]=f^\prime g + fg^\prime
\end{equation}
\begin{equation}
    \frac{d}{dx}\lbrack\frac{f(x)}{g(x)}\rbrack=\frac{f^\prime(x)g(x)-f(x)g^\prime(x)}{[g(x)]^2}
\end{equation}
\begin{equation}
    \frac{d}{dx}\lbrack f(g(x))\rbrack = f^\prime(g(x))g^\prime(x)
\end{equation}
\hformbar

%%% MEAN VALUE THEOREM --------------------------------------------------------------
\formdesc{Mean Value Theorem}
\footnotesize{If $f$ is a continuous function over the closed interval $\lbrack a, b\rbrack$ differentiable over the open interval $(a, b)$, and $c\in(a, b)$}.
\begin{equation}
\label{mvt}
f^\prime(c) = \frac{f(b)-f(a)}{b-a}
\end{equation}
\hformbar


%%% Enumeration example -----------------------------------------------------------
%%%\begin{enumerate}
%%% \item decide on the number of classes k where \(2^k > n\)
%%% \item  determine the class interval i by \(i\geq\frac{\text{max value - min value}}{k}\)
%%% \item set individual class limits.
%%% \item tally the values into the classes.
%%% \item count the number of items in each class.
%%%\end{enumerate}

%%% DERIVATIVES AND THE SHAPE OF A GRAPH ----------------------------------------------
\formdesc{Derivatives and the Shape of a Graph}
\begin{enumerate}
    \item If $f^\prime$ changes sign from positive when $x < c$ to negative when $x > c$,     then $f(c)$ is a local maximum of $f$.
    \item If $f^\prime$ changes sign from negative when $x < c$ to positive when $x > c$,     then $f(c)$ is a local minimum of $f$.
    \item If $f^\prime$ has the same sign for $x < c$ and $x > c$, then $f(c)$ is neither     a local maximum nor a local minimum of $f$.
\end{enumerate}
\hformbar

%%% CONCAVITY AND POINTS OF INFLECTION
\formdesc{Concavity and Points of Inflection}
\begin{center}
\begin{tabular}{||c | c | c | c||} 
 \hline
 Sign of $f^\prime$ & Sign of $f^{\prime\prime}$ & Is $f$ inc. or dec.? & Concavity \\ [0.5ex] 
 \hline\hline
 + & + & Increasing & Up \\ 
 \hline
 + & - & Increasing & Down \\
 \hline
 - & + & Decreasing & Up \\
 \hline
 - & - & Decreasing & Down \\
 \hline
\end{tabular}
\end{center}
\hformbar

%%% POWER RULE FOR INTEGRATION
\formdesc{Power Rule of Integration}
\begin{equation}
\label{integration power rule}
\int x^n \space dx = \frac{x^{n+1}}{n+1}+C
\end{equation}
\hformbar

%%% RIEMANN SUMS
\formdesc{Riemann Sums}
\begin{equation}
\label{Left/Right Hand Riemann Sums}
L_n \approx \sum^n_{i=1}f(x_{i-1})\Delta x
\end{equation}

\begin{equation}
R_n \approx \sum^n_{i=1}f(x_i)\Delta x
\end{equation}

\begin{equation}
A\approx \sum^n_{i=1}f(x^*_i)\Delta x
\end{equation}
\hformbar

%%% THE DEFINITE INTEGRAL
\formdesc{The Definite Integral}
\begin{equation}
    \int_a^b f(x)\space dx=\lim_{n\rightarrow \infin}\sum^n_{i=1}f(x^*_i)\Delta x
\end{equation}
where:
\begin{itemize}
    \item The limit exists and $f(x)$ is continuous on $\lbrack a, b\rbrack$, then $f$ is integrable on $\lbrack a, b\rbrack$.
\end{itemize}
\hformbar

%%% FIND AVG OF A FUNCTION
\formdesc{Average of a Function}
\footnotesize{If $f(x)$ is continuous over the interval $\lbrack a, b\rbrack$.}
\begin{equation}
    \bar{f}=\frac{1}{b-a}\int_a^b f(x)\space dx
\end{equation}
\hformbar

%%% FERMAT'S THEOREM
\formdesc{Fermat's Theorem}
\footnotesize{If $f$ has a local extremum at $c$ and $f$ is differentiable at $c$, then}
\begin{equation}
    f^\prime(c) = 0
\end{equation}
\hformbar

%%% Linear Approximators
\formdesc{Linear Approximations}
\footnotesize{Consider a differentiable function $f$ such that $\lim_{x\rightarrow a}f(x)=0$. For $x$ near $a$, we can write}
\begin{equation}
    L(x) = f(a)+f^\prime(a)(x-a)
\end{equation}
\hformbar

\formdesc{Common Derivatives and Their Antiderivatives}
\begin{center}
\begin{tabular}{||c | c||} 
 \hline
 Differentiation & Indefinite Integration \\ [4ex] 
 \hline\hline
 $\frac{d}{dx}(k)=0$ & $\int k\space dx=kx+C$ \\ [2ex]
 \hline
 $\frac{d}{dx}(x^n)=nx^{n-1}$ & $\int x^n\space dx = \frac{x^{n+1}}{n+1}+C$ \\ [2ex]
 \hline
 $\frac{d}{dx}(\ln |x|)=\frac{1}{x}$ & $\int\frac{1}{x}\space dx=\ln |x| + C$ \\ [2ex]
 \hline
 $\frac{d}{dx}(e^x)=e^x$ & $\int e^x\space dx=e^x + C$ \\ [2ex]
 \hline
 $\frac{d}{dx}(\sin x)=\cos x$ & $\int\cos x\space dx=\sin x + C$ \\ [2ex] 
 \hline
 $\frac{d}{dx}(\cos x)=-\sin x$ & $\int\sin x\space dx=-\cos x + C$ \\ [2ex]
 \hline
 $\frac{d}{dx}(\tan x)=\sec^2 x$ & $\int\sec^2 x\space dx = \tan x + C$ \\ [2ex]
 \hline
 $\frac{d}{dx}(\cot x)=-\csc^2 x$ & $\int\csc^2 x\space dx=-\cot x + C$ \\ [2ex]
 \hline
 $\frac{d}{dx}(\sec x)=\sec x\tan x$ & $\int\tan x\sec x\space dx=\sec x + C$ \\ [2ex]
 \hline
 $\frac{d}{dx}(\csc x)=-\csc x\cot x$ & $\int\cot x\csc x\space dx=-\csc x + C$ \\ [2ex]
 \hline
 $\frac{d}{dx}(\arcsin x)=\frac{1}{\sqrt{1-x^2}}$ & $\int\frac{1}{\sqrt{1-x^2}}\space dx = \arcsin x + C$ \\ [2ex]
 \hline
  $\frac{d}{dx}(\arccos x)=-\frac{1}{\sqrt{1-x^2}}$ & $\int -\frac{1}{\sqrt{1-x^2}}\space dx = \arccos x + C$ \\ [2ex]
 \hline
  $\frac{d}{dx}(\arctan x)=\frac{1}{1+x^2}$ & $\int\frac{1}{1+x^2}\space dx = \arctan x + C$ \\ [2ex]
 \hline
 
\end{tabular}
\end{center}
\hformbar

\end{document}